\documentclass[9pt,a4paper,twocolumn]{jsarticle}

%%%%%%%%% 利用パッケージ一覧 %%%%%%%%%%%

% 原稿の余白を調整するためのパッケージ
\usepackage[truedimen, top=20truemm, bottom=20truemm, left=20truemm, right=20truemm]{geometry} 

% 原稿のレイアウトを確認するためのパッケージ
\usepackage{layout}

% フォントの設定用のパッケージ
\usepackage[T1]{fontenc}       %Latin Modern
\usepackage{lmodern}           %Latin Modern

% 数式用のパッケージ
\usepackage{amsmath,amssymb}
\usepackage{bm}               % 太文字ベクトル用

% 図の描画用のパッケージ
\usepackage[dvipdfmx]{graphicx}

% 図のキャプションの設定用のパッケージ
\usepackage[labelsep=space, figurename=Fig.\;, tablename=Table\;]{caption}
\usepackage[labelformat=parens,subrefformat=parens]{subcaption}

% 複数行にわたるコメントアウトのためのパッケージ
\usepackage{comment}

% URLを扱うためのパッケージ
\usepackage{url}

%%%%%%%% レイアウト設定 %%%%%%%%%%%
\pagestyle{empty}
\abovecaptionskip=4truept   
\belowcaptionskip=-3truept
\setlength{\columnsep}{2zw} % 2段組の間の間隔

% 参考文献のスタイルの設定
\makeatletter
\renewcommand{\@biblabel}[1]{(#1)\hspace{6truept}}
\makeatother

% 参考文献リストの参照のスタイルの設定
\makeatletter
\def\@cite#1{\hspace{1truept}$^{\scalebox{0.65}{#1)}}$}
\makeatother

% 章のスタイルの設定
\renewcommand{\thesection}{\arabic{section}.\hspace{-15truept}}

% urlのフォントの設定
\renewcommand\UrlFont{\rmfamily}

%%%%%%%%%%%%%%%%%%ドキュメント開始%%%%%%%%%%%%%%%%%%%%%
\begin{document}

%%%%%%%%%%%% タイトル %%%%%%%%%%%%%
\twocolumn[
\vspace{13pt}
\centerline{\LARGE \textsf{[A00]}%
\textgt{卒業研究題目を入れる}}
\vspace{2.0\baselineskip}
\rightline{I\hspace{-1pt}I\hspace{-1pt}I類ほげほげ%
プログラム ほにゃらら研究室}

\rightline{1700000 UEC 太郎}
\vspace{2.0\baselineskip}
]
%%%%%%%%%%%%% 本文開始 %%%%%%%%%%%%%%%%

\section{序論}

\section{結論}

\bibliographystyle{junsrt}
\begin{thebibliography}{9}
\bibitem{cite:label}
T. Yoshikawa: Manipulability of Robotic Mechanisms, The International Journal of Robotics Research, Vol. 4, pp. 3-9,
(1985).
\end{thebibliography}
\end{document}
